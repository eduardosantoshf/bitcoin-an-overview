\documentclass{article}
% Margin definition.
\usepackage[a4paper,total={6.8in, 8.5in}]{geometry}
\usepackage{parskip}
% Images.
\usepackage{graphicx}
\usepackage[hidelinks, bookmarks=true]{hyperref}
\usepackage{float}
% Encoding.
\usepackage[english]{babel}
\usepackage[utf8]{inputenc}
% Allow multiline comments
\usepackage{verbatim} 
% Helvetic font.
\usepackage[scaled]{helvet}
\renewcommand\familydefault{\sfdefault} 
% Header for UA logo.
\usepackage{fancyhdr}
% package for plots / graphics
\usepackage{pgfplots}
\pgfplotsset{width=10cm,compat=newest}
% Dots in index.
\usepackage[titles]{tocloft}
\renewcommand{\cftsubsecleader}{\Large\cftdotfill{0}}
\renewcommand{\cftsecleader}{\Large\cftdotfill{0}}
\renewcommand{\cftsecfont}{\large\bfseries\scshape}
\renewcommand{\cftsubsecfont}{\scshape}
\renewcommand*{\HyperDestNameFilter}[1]{\jobname-#1}
% Dot after number in (sub)sections and in toc.
\renewcommand{\cftsecaftersnum}{.}
\renewcommand{\cftsubsecaftersnum}{.}
\usepackage{titlesec}
\titlelabel{\hspace{-0.5cm}\quad}
\usepackage[letterspace=45]{microtype}
\newcommand*{\fullref}[1]{\hyperref[{#1}]{\autoref*{#1} \nameref*{#1}}}
% Header with UA logo definition. 
\pagestyle{fancy}
\fancyhf{}
\chead{
    \includegraphics[width=5in]{./images/header_ua.png}
}
\setlength\headheight{20pt}
% Footer with page number.
\rfoot{Page \thepage}
\renewcommand{\footrulewidth}{0.1pt}
% Rename table of contents title to "Index"
\renewcommand{\contentsname}{\normalsize Index \vspace{0.6cm}}
% Add text with hyperlink
\usepackage{hyperref}
%\hypersetup{
%    colorlinks=true,
%    linkcolor=blue,
%    filecolor=magenta
%}
% Water mark
\newsavebox\mybox
\usepackage[printwatermark]{xwatermark}
\usepackage{xcolor}
\usepackage{tikz}
% paragraph
\newcommand\tab[1][1cm]{\hspace*{#1}}
\setlength\parindent{24pt}
%images
 \usepackage{graphicx}
\usepackage{caption}
% footnotes at bottom
\usepackage[bottom]{footmisc}
% Urls with line break
\usepackage{pdflscape}
% Drawing functions
\usepackage{tikz}
\usepackage{pgfplots}
\pgfplotsset{width=7cm, height=4cm, compat=1.17}

\usepackage{multicol}
\setlength{\columnsep}{1cm}

%%%%%%%%%%%% References/Bibliography %%%%%%%%%%%%
\usepackage{biblatex}
\addbibresource{bibliography.bib}

%%%%%%%%%%%%%%%%%%%%%%%%%%%%%%%%%%%%%%%%%%%%%%%%%

\begin{document}

%%%%%%%%%%%%%%%%%% Cover Page %%%%%%%%%%%%%%%%%%
\title{\vspace{-0.9cm}
       \vspace{1cm}
       \normalsize
       \raggedright\textbf{Title: \hspace{1.5cm} Bitcoin: An Overview} \\ \vspace{0.4cm}
       \raggedright\textbf{Authors: \hspace{0.95cm} Eduardo Santos, Hugo Ferreira, João Soares, Pedro Bastos} \\ \vspace{0.4cm}
       \raggedright\textbf{Date: \hspace{1.45cm} 13/04/2021} \\}
\author{}
\date{}

\maketitle
\thispagestyle{fancy}

%%%%%%%%%%%%%%%%%% END Cover Page %%%%%%%%%%%%%%%%%%

\vspace{-1.4cm}

\tableofcontents


\fontsize{10pt}{13pt}
\selectfont
\lsstyle

\titlelabel{\thetitle.\quad}	

\newpage

\section{Introductory Note}

\tab This report consists of a brief analysis of Bitcoin and its protocol, exploring its surge and growth in the cryptocurrency market.
We will also address the topic of Blockchain, and how does it work.

\section{Summary / Abstract}

\tab Bitcoin is a cryptocurrency that has been widely spoken over the past few years, majorly because of what it represents: the beginning if the globalization of digital currencies.

\section{Blockchain}

\subsection{What is the Blockchain?}

\subsection{How does it work?}



\section{Bitcoin}

\subsection{What is Bitcoin?}

\subsection{The Bitcoin Protocol}

\tab Being a decentralized digital currency, Bitcoin is not something that we can physically own, which in itself generates a large deviation from the concept of money that we are used to and that we use in our day-to-day.

To use this digital asset, there are many steps/rules that need to be followed, they make up the Bitcoin protocol.

\subsubsection{How does it work?}

\tab Bitcoins are not stored centrally nor locally, they exist in a distributed ledger called blockchain. This ledger runs on a P2P\footnote{peer-to-peer} network of computers. 

Owning a Bitcoin is, nothing more, nothing less than simply having the power to transfer it to someone else, with the transaction being recorded in the blockchain. But how does this work?

This is made possible with the usage of an private key - public key ECDSA\footnote{Elliptic Curve Digital Signature Algorithm} pair.

\subsubsection{ECDSA}

\tab ECDSA is a variant of the DSA\footnote{Digital Signature Algorithm}, this uses elliptic curve cryptography. An elliptic curve is mathematically represented by the following equation:

\[y^2 = ax^3 + bx^2 + cx + d\]

In the case of Bitcoin:

\[a = 1\]
\[b = 0\]
\[c = 0\]
\[d = 7\: mod\: 1.158 \times 10^{77}\]

Replacing the values, we obtain:

\[y^2 = x^3 + 7 \bmod{1.158 \times 10^{77}}\]

Finally, this equation can be translated to the following graphic:

\centering
\begin{tikzpicture}
    \begin{axis}[
        width=0.5\textwidth,
        height=0.3\textwidth,
        xmin=-5,
        xmax=5,
        ymin=-5,
        ymax=5,
        xlabel={$x$},
        ylabel={$f(x)$},
        scale only axis,
        axis lines=middle,
        smooth,
        domain=-1.912931:3,
        xtick={-5,...,0,...,5},
        ytick={-5,...,0,...,5},
    ]
        \addplot [red] {sqrt(x^3 + mod(7, 1.158*10^(77)))};
        \addplot [red] {-sqrt(x^3 + mod(7, 1.158*10^(77)))};
    \end{axis}
\end{tikzpicture}

% Add "References" to table of contents
\addcontentsline{toc}{section}{References}
% No cite makes all references appear, even if there's no citation on the text
\nocite{*}
\printbibliography

\end{document}