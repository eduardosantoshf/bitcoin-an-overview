\documentclass{article}
% Margin definition.
\usepackage[a4paper,total={6.8in, 8.5in}]{geometry}
\usepackage{parskip}
% Images.
\usepackage{graphicx}
\usepackage[hidelinks, bookmarks=true]{hyperref}
\usepackage{float}
% Encoding.
\usepackage[english]{babel}
\usepackage[utf8]{inputenc}
\usepackage{epigraph}
% Allow multiline comments
\usepackage{verbatim} 
% To have another layer of sub sections - \paragraph
\usepackage{titlesec}

\setcounter{secnumdepth}{4}

\titleformat{\paragraph}
{\normalfont\normalsize\bfseries}{\theparagraph}{1em}{}
\titlespacing*{\paragraph}
{0pt}{3.25ex plus 1ex minus .2ex}{1.5ex plus .2ex}
% Helvetic font.
\usepackage[scaled]{helvet}
\renewcommand\familydefault{\sfdefault} 
% Header for UA logo.
\usepackage{fancyhdr}
% package for plots / graphics
\usepackage{pgfplots}
\pgfplotsset{width=10cm,compat=newest}
% Dots in index.
\usepackage[titles]{tocloft}
\renewcommand{\cftsubsecleader}{\Large\cftdotfill{0}}
\renewcommand{\cftsecleader}{\Large\cftdotfill{0}}
\renewcommand{\cftsecfont}{\large\bfseries\scshape}
\renewcommand{\cftsubsecfont}{\scshape}
\renewcommand*{\HyperDestNameFilter}[1]{\jobname-#1}
% Dot after number in (sub)sections and in toc.
\renewcommand{\cftsecaftersnum}{.}
\renewcommand{\cftsubsecaftersnum}{.}
\usepackage{titlesec}
\titlelabel{\hspace{-0.5cm}\quad}
\usepackage[letterspace=45]{microtype}
\newcommand*{\fullref}[1]{\hyperref[{#1}]{\autoref*{#1} \nameref*{#1}}}
% Header with UA logo definition. 
\pagestyle{fancy}
\fancyhf{}
\chead{
    \includegraphics[width=5in]{./images/header_ua.png}
}
\setlength\headheight{20pt}
% Footer with page number.
\rfoot{Page \thepage}
\renewcommand{\footrulewidth}{0.1pt}
% Rename table of contents title to "Index"
\renewcommand{\contentsname}{\normalsize Index \vspace{0.6cm}}
% Add text with hyperlink
\usepackage{hyperref}
%\hypersetup{
%    colorlinks=true,
%    linkcolor=blue,
%    filecolor=magenta
%}
% Water mark
\newsavebox\mybox
\usepackage[printwatermark]{xwatermark}
\usepackage{xcolor}
\usepackage{tikz}
% paragraph
\newcommand\tab[1][1cm]{\hspace*{#1}}
\setlength\parindent{24pt}
%images
 \usepackage{graphicx}
\usepackage{caption}
% footnotes at bottom
\usepackage[bottom]{footmisc}
% Urls with line break
\usepackage{pdflscape}
% Drawing functions
\usepackage{tikz}
\usepackage{pgfplots}
\pgfplotsset{width=7cm, height=4cm, compat=1.17}

\usepackage{multicol}
\setlength{\columnsep}{1cm}

%%%%%%%%%%%% References/Bibliography %%%%%%%%%%%%
\usepackage{biblatex}
\addbibresource{bibliography.bib}

%%%%%%%%%%%%%%%%%%%%%%%%%%%%%%%%%%%%%%%%%%%%%%%%%

\begin{document}

%%%%%%%%%%%%%%%%%% Cover Page %%%%%%%%%%%%%%%%%%
\title{\vspace{-0.9cm}
       \vspace{1cm}
       \normalsize
       \raggedright\textbf{Title: \hspace{1.5cm} Bitcoin: An Overview} \\ \vspace{0.4cm}
       \raggedright\textbf{Authors: \hspace{0.95cm} Eduardo Santos, Hugo Ferreira, João Soares, Pedro Bastos} \\ \vspace{0.4cm}
       \raggedright\textbf{Date: \hspace{1.45cm} 13/04/2021} \\}
\author{}
\date{}

\maketitle
\thispagestyle{fancy}

%%%%%%%%%%%%%%%%%% END Cover Page %%%%%%%%%%%%%%%%%%

\vspace{-1.4cm}

\tableofcontents


\fontsize{10pt}{13pt}
\selectfont
\lsstyle

\titlelabel{\thetitle.\quad}	

\newpage

\section{Introductory Note}

\tab This report consists of a brief analysis of Bitcoin and its protocol, exploring its surge and growth in the cryptocurrency market.
We will also address the topic of Blockchain, and how does it work.

\section{Summary / Abstract}

\tab Bitcoin is a cryptocurrency that has been widely spoken over the past few years, majorly because of what it represents: the beginning if the globalization of digital currencies.

\section{Cryptocurrencies}

\subsection{Definition of Cryptocurrency}

\subsubsection{What is Money}

\tab To understand what is a cryptocurrency, we first need to understand the concept of money as a form of currency, as well as a medium of exchange of goods and services. Before money as we know it existed, the payment/trade of goods and services was made with commodity money.

\paragraph{Commodity Money}
 
\tab Commodity money is a physical good whose value comes from the resource of which it is made, in other words, that has “intrinsic value” - utilization outside of its use as money. As such its underlying value and use ensures that people trust it because it has value in and of itself and so people can trade it freely with the knowledge that someone will accept it. This type of money is usually durable, divisible, easily exchangeable, and rare.

Some historic examples of commodity money being salt (extremely important to conserve food) and tobacco - after World War II some parts of Europe briefly used this type of money.

Eventually, society evolved into a system of representative money and fiat money.

\paragraph{Representative Money}

\tab Representative money is a form of currency often printed on paper that represents something of value - a resource/commodity - but has little or no value of its own. Its acceptance requires that the population trusts the certificate as much as the value that it represents. It was usually backed by a physical resource such as precious metals like gold or silver that existed in the United States as silver certificates or gold certificates issued by state banks. Nowadays, financial instruments like checks and credit cards are the most common forms of representative money.

%Added [H] after so that image is placed after the text above
%This is only possible by adding also \usepackage{float} to the preamble
\begin{figure}[H]
    \begin{center}
        \includegraphics[width=0.3\textwidth]{images/representative_money.png}
        \caption{\textit{Representative money}}
    \end{center}
\end{figure}

\paragraph{Fiat Money}

\tab Fiat money is the most common type of currency nowadays, with every other nation operating on some form of fiat money like the Euro or the GBP\footnote{Great British Pound}.

\renewcommand{\epigraphflush}{center}
\epigraph{\textbf{Fiat money is a currency without an underlying value. Instead, its value is derived by government and the trust people place in its value. In other words, it is a form of currency that only holds value because of government enforcement.}}{\textit{Source:\href{https://boycewire.com/fiat-money-definition/#FiatVsRepresentative}{\underline{BoyceWire's definition of fiat money}}}}

It’s a type of money where trust is fundamental because, although its value is given by an official law or order, its true value lies in the trust that people place in it. If consumers and businesses did not trust its enforcement value, then it would not be accepted as a method of payment/trade of goods and services. It is only because people believe others will accept it as a method of payment/trade that it maintains its value, if this did not happened, it would be worthless.

\begin{figure}[H]
    \begin{center}
        \includegraphics[width=0.3\textwidth]{images/fiat_money.png}
        \caption{\textit{Fiat money}}
    \end{center}
\end{figure}

Some advantages of this type of money is that it can technically be unlimited and it's cheaper to produce, central banks can "print" as much money as they want, and the production cost is small if not null. Also, much of the transactions are being done online nowadays.

On the other hand, gold, silver, or any other resource/commodity is limited by the extraction process, for instance on a mine, and the limited nature of its existence with an exorbitant cost, because it requires workers to mine it, process it, transport it, and then finally store it in a safe place.

The two previously mentioned advantages create stability because the money supply can react quicker to an increasing/decreasing economic output and enter the market in a short period of time, preventing and decreasing the effects of cases like the Great Depression by creating a greater level of price stability, in other words controlling high inflation and deflation.

This stability was demonstrated in the 2008 financial crisis where prices remained relatively stable with inflation rising by an average of over 1,5 percent in the preceding three years, something that most economists favor while in the Great Depression inflation declined by an average of over 8,8 percent in the preceding three years.

However, stability relies more on the decisions made by the central banks that can have more of an effect than the type of money or even the type of currency used, as demonstrated by the Venezuela currency, the Bolivar Fuerte, replaced the original Bolivar in 2008, but inflation is still very high because among other factors there are no clear restrictions on how much the government can or cannot print money.

\begin{figure}[H]
    \begin{center}
        \includegraphics[width=0.5\textwidth]{images/venezuela_inflation.png}
        \caption{\href{https://ichef.bbci.co.uk/news/640/cpsprodpb/7822/production/_105345703_venezuela-inflation_v3_976-nc.png}{\underline{Venezuela's inflation fluctuation between 2000 and 2018}}}
    \end{center}
\end{figure}

\subsubsection{What is Cryptocurrency}

A cryptocurrency is a digital currency, in other words, a medium of exchange that is encrypted and decentralized. Unlike the U.S. Dollar or the Euro, there is no central authority that manages and maintains the value of a cryptocurrency. It doesn't rely on banks to verify transactions, instead it's a P2P\footnote{peer-to-peer} system that can enable anyone anywhere that has a digital wallet to send and receive payments using a technology called blockchain, which we will talk about later on in this report. 

A cryptocurrency wallet doesn't actually hold any currency, it merely provides an address for your funds.

Unlike physical money that is carried around and exchanged, cryptocurrency payments exist purely as digital entries to an online database that describes specific transactions. When you transfer cryptocurrency funds, the transactions are recorded in a public ledger.

It's possible to buy or sell cryptocurrency in exchange for a fiat currency like the U.S. Dollar using a cryptocurrency exchange. Exchanges, which can hold deposits in both fiat and cryptocurrencies, credit and debit the appropriate balances of buyers and sellers in order to complete cryptocurrency transactions. People can also use cryptocurrency to buy  products and services with more and more companies accepting this type of currencies as a payment method.

Cryptocurrency got its name because it uses encryption to provide security and safety in transactions. This means advanced coding is involved in storing and transmitting cryptocurrency data between wallets and to public ledgers. 

\begin{figure}[H]
    \begin{center}
        \includegraphics[width=0.5\textwidth]{images/types_of_cryptocurrencies.png}
        \caption{\href{https://coinratecap.com/public/storage/posts/July2019/5aGA2VGhWLCMy27A49my.png}{\underline{Types of cryptocurrencies}}}
    \end{center}
\end{figure}

\subsubsection{Comparison between Cryptocurrencies and Fiat Money}

\tab Compared to fiat money, cryptocurrencies share some characteristics as both can be used for payments and as a store of value. Both rely on widespread trust of people in order to function as a means of exchange although cryptocurrencies' trust and value is based on the underlying technology blockchain while fiat money's value and trust is derived from a governmental institution or a trustworthy authority such as the ECB\footnote{European Central Bank}.

Some key differences are that cryptocurrency is  produced and distributed through a process called mining and most have a cap which means there is a set amount of coins that will ever be in supply, making it possible to tell the amount in circulation at any given time, something impossible in fiat money.
 
Cryptocurrency is not controlled by a centralised authority, transactions can’t be reversed, cancelled or charged back and while Fiat Money can be physical, cryptocurrency is only digital.

\begin{figure}[H]
    \begin{center}
        \includegraphics[width=0.8\textwidth]{images/cryptocurrencies_vs_fiat_money.jpeg}
        \caption{\href{https://bitpanda-academy.imgix.net/null94a7b99f-2399-48fa-a503-876e2ba6f2bf/Bitpanda-Infographics_2-bitcoin_fiat.png?auto=compress\%2Cformat&fit=min&fm=jpg&q=80&w=2100}{\underline{Difference between cryptocurrencies and fiat money}}}
    \end{center}
\end{figure}

\subsection{How Cryptocurrencies Work}

\section{Blockchain}

\subsection{What is the blockchain}

\tab A blockchain is a chain of blocks that contains information. A blockchain is a distributed ledger that is completely open to anyone, and it relies on a key property: when some data is recorded inside a blockchain, it becomes very difficult to change it. But how does that work?

Each block in the blockchain contains:

\begin{itemize}
    \item \textbf{Data} - this stored data depends on the type of blockchain. In the case of the Bitcoin blockchain, it stores the following details:
    \begin{itemize}
        \item Sender
        \item Receiver
        \item Amount of coins to be traded
    \end{itemize}
    \item \textbf{Hash of the block} - it works like a fingerprint, identifying the block and all of its contents and it is always unique.
    \item \textbf{Hash of the previous block} - this creates a chain of blocks, they form the blockchain itself.
\end{itemize}

Once each block is created, its hash is calculated. Changing something inside the block, will cause the hash to change. In another words, hashes are very useful when we want to detect changes to blocks, if the fingerprint of a block changes, it no longer is the same block.

\subsection{How does it work?}

\tab Let's take, for instance, the chain of blocks in the image below:

\begin{figure}[H]
    \begin{center}
        \includegraphics[width=0.6\textwidth]{images/changes_in_block.png}
        \caption{\textit{Example of a blockchain}}
    \end{center}
\end{figure}

As we can see, each block has a hash and the hash of the previous block. So block number 2 points to block number 1, and block number 3 points to block number 2. The first block is called the \textit{Genesis Block}, as it has not got previous blocks.

Now, let's say that someone tampers with the second block, as shown in the image below:

\begin{figure}[H]
    \begin{center}
        \includegraphics[width=0.6\textwidth]{images/invalid_hash.png}
        \caption{\textit{Example of a tampered blockchain}}
    \end{center}
\end{figure}

This will cause the hash of the block to change as well. This change will make block 3 and all following blocks invalid because they no longer store a valid hash of the previous block. So changing a a single block will make all following blocks invalid.

But using hashes is not enough to prevent tampering. Computers these days are very fast and can calculate hundreds of thousands of hashes per second, this being said, we could tamper with a block, and recalculate all the hashes of the previous blocks to make our blockchain valid again.

To mitigate this, blockchains have something called proof-of-work.

\subsubsection{Proof-of-work}

\tab Proof-of-work is a mechanism that slows down the creation of new blocks. In Bitcoin's case, it takes about ten minutes to calculate the required proof-of-work and add a new block to the chain. This mechanism makes it very hard to tamper with the blocks, because if we tamper with one block, we will need to recalculate the proof-of-work for all the following blocks.

So, the security of a blockchain comes from its creative use hashing and the proof-of-work mechanism.

But there is one more way that blockchains secure themselves, and that is by being distributed. Instead of using a central entity to manage the chain, blockchains use a P2P network that anyone can join. When someone joins this network, he gets the full copy of the blockchain, the node can use this to verify that everything is still in order.

But what specifically happens when someone creates a new block? That block is sent to everyone on the network, each node then verifies the block, to make sure that it has not been tampered with. If everything checks out, each node adds this block to their own blockchain. 

All the nodes in this network create consensus: they agree about what blocks are valid and which are not. Blocks that are tampered with will be rejected by other nodes in the network. This being said, to successfully tamper with a blockchain we would need to tamper with all blocks on the chain, redo the proof-of-work for each block, and take control of more than 50\% of the P2P network. Only then our tampered block would be accepted by everyone else. This entire process is almost impossible to do!

\subsubsection{Smart contracts}

\tab Blockchains are also constantly evolving. One of the most recently developments is the creation of smart contracts.

These contracts are simple programs that are stored on the blockchain and can be used to automatically exchange coins, based on certain conditions.

\section{Bitcoin}

\subsection{What is Bitcoin}

\subsection{The Seven Pillars of Bitcoin}

\tab The Bitcoin system has seven main pillars that allow the construction of a fair market. 

\begin{figure}[H]
    \begin{center}
        \includegraphics[width=\textwidth]{images/pillars.png}
        \caption{\textit{Pillars of Bitcoin}}
    \end{center}
\end{figure}

\subsubsection{Open Source}

\tab This will be the first pillar approached. Bitcoin is based on the open source software core. This means that the source code is available to anyone who wants to see how the network works. Not only that, anyone can contribute to it. There is a resemblance with the Linux community. The community that daily keeps improving it is its main strength.

With that in mind, bitcoin will not be replaced by a more powerful technology. As mentioned earlier, its developers will keep improving it, making it evolving alongside the internet. Besides that, its security is an essential point to keep improving. And again, thanks to its developers, it will keep up, allowing bitcoin to continue its dominance in the cryptocurrencies market.

\subsubsection{Transparent}

\tab As it was explained in the open source topic, bitcoin blockchain is based on open source code and anyone can check it. Besides that, anyone can enter the network, making bitcoin trustless and permissionless. Anyone can verify any transaction made in the network. This allows people to form their own opinion regarding this market. That is why bitcoin is originally based on a simple quote:

\renewcommand{\epigraphflush}{center}
\epigraph{\textbf{Don't trust, verify.}}{\textit{Bitcoin's motto}}

This is a very important aspect of bitcoin. All users can verify everything by their own, making their decisions a lot easier and allowing them the opportunity to see the impact of every transaction.


\subsubsection{Neutral}

\tab As this system, capable of revolutionizing the current financial system, is available to literally anyone, the success of bitcoin its every user's responsibility. And by this we mean that the users are the ones who make the difference. So, there isn't really an owner of bitcoin. Everyone has equally the same influence. With this, your transactions are only made by you and no one can have a say in it. 

There are many studies that point bitcoin to the U.S dollar's next replacement as the world's reserve currency. They predict the fall of the U.S dollar and the rise of bitcoin, as it is politically neutral. 

%Added [H] after so that image is placed after the text above
%This is only possible by adding also \usepackage{float} to the preamble
\begin{figure}[H]
    \begin{center}
        \includegraphics[width=0.5\textwidth]{images/dollar_bitcoin.jpeg}
    \end{center}
\end{figure}

\subsubsection{Decentralized}

\tab The whole bitcoin system is decentralized. This means that it is distributed and resistant to potential attacks. Because of this, its up-time is incredibly high: \textbf{99.985\%}, making it almost unbreakable. 

Bitcoin remains the easiest, fastest and most efficient solution to transfer any amount of money across the world. When compared to other solutions, like banks, its transaction fees and taxes are almost insignificant. In addition, the transaction time is also much faster, completing any transaction in 10 minutes. This is usually 300 times faster than any bank transaction. 

As you probably go to this, it is pretty obvious that bitcoin allows you to make transactions without the inter bank system making it difficult.

\subsubsection{Censorship-Resistant}

\tab As we discussed before, bitcoin has always been a democracy in the sense that there is no leader, all users are equally relevant. Each users possesses their own bitcoins as long as they have their private keys, and no one can take that away. 

\renewcommand{\epigraphflush}{center}
\epigraph{\textbf{Not your keys, not your Bitcoins.}}{\textit{Bitcoin's rules}}

Therefore, you have no fear that your bitcoins will be confiscated by any government. This is a major guarantee that you can have by owning bitcoins.

\subsubsection{Secure}

\tab Bitcoin miners are responsible for validating blocks of transactions by allowing their own computing power to be used by the network. To ensure that these miners run the operation smoothly, they are rewarded by 2 things:

\begin{itemize}
    \item \textbf{Halving - Bitcoin reward every 210,000 blocks issued}
    \item \textbf{Transaction fees}
\end{itemize}

 Currently, this makes bitcoin the most secure network in the entire world. 
 
\subsubsection{Monetary Policy}

\tab The bitcoin's policy will prevent bitcoin from its vanishment. Unlike the U.S dollar, bitcoin exists in limited amounts, as there will be no more than 21 million BTC\footnote{Bitcoins} in circulation. This ensures users that they always own the same percentage of the world's bitcoins as time goes by.

In addition, the process of creating new bitcoins is always predictable. No one can change it and everyone can know when new bitcoin will be created, as it is not influenced by any human decision. 

This makes bitcoin's policy better than the one practiced by the banks. The banks' policies change the money quantity and this is not a problem with bitcoin.

\subsection{The Bitcoin Protocol}

\tab Being a decentralized digital currency, Bitcoin is not something that we can physically own, which in itself generates a large deviation from the concept of money that we are used to and that we use in our day-to-day. To use this digital asset, there are many steps/rules that need to be followed, they make up the Bitcoin protocol.

Bitcoins are not stored centrally nor locally, they exist in a distributed ledger called blockchain. This ledger runs on a P2P network of computers. 

Owning a Bitcoin is, nothing more, nothing less than simply having the power to transfer it to someone else, with the transaction being recorded in the blockchain. But how does this work?

This is made possible with the usage of an private key - public key ECDSA\footnote{Elliptic Curve Digital Signature Algorithm} pair.

\subsubsection{ECDSA}

\tab ECDSA is a variant of the DSA\footnote{Digital Signature Algorithm}, this uses elliptic curve cryptography. An elliptic curve is mathematically represented by the following equation:

\[y^2 = ax^3 + bx + c\]

In the case of Bitcoin:

\[a = 1\]
\[b = 0\]
\[c = d\: mod\: p\]
\[d = 7\]
\[p = 1.158 \times 10^{77}\]

Replacing the values, we obtain:

\[y^2 = x^3 + 7 \bmod{1.158 \times 10^{77}}\]

Finally, this equation can be translated to the following graphic:

%\vspace{5mm} %5mm vertical space

%Added [H] after so that image is placed after the text above
%This is only possible by adding also \usepackage{float} to the preamble
\begin{figure}[H]
    \begin{center}
        \includegraphics[width=0.5 \textwidth]{images/Kobiltz_curve.png}
        \caption{\textit{Kobiltz curve}}
    \end{center}
\end{figure}

%\vspace{5mm} %5mm vertical space

\begin{comment}
%Added [H] after so that image is placed after the text above
%This is only possible by adding also \usepackage{float} to the preamble
\begin{figure}[H]
    \begin{center}
        \begin{tikzpicture}
            \begin{axis}[
                width=0.5\textwidth,
                height=0.3\textwidth,
                xmin=-5,
                xmax=5,
                ymin=-5,
                ymax=5,
                xlabel={$x$},
                ylabel={$f(x)$},
                scale only axis,
                axis lines=middle,
                smooth,
                domain=-1.912931:3, %supostamente, sem esta linha o gráfico divide-se em 2 partes
                xtick={-5,...,0,...,5},
                ytick={-5,...,0,...,5},
            ]
                \addplot [red] {sqrt(x^3 + mod(7, 1.158*10^(77)))};
                \addplot [red] {-sqrt(x^3 + mod(7, 1.158*10^(77)))};
            \end{axis}
        \end{tikzpicture}
    \end{center}
    \caption{Kobiltz curve}
\end{figure}
\end{comment}

The curve on the graph above is called \textit{Kobiltz curve}, and its often referred as \textit{secp256k1}. This curve was created on purpose, and with a very defined objective: to increase efficiency in the generation of public and private keys, making the computation of this key pair 30\% faster. But this is not the only thing that differentiates this curve from the rest of the elliptical curves, due to the precision in choosing the values of the constants \(a, b, c\) and \(d\), the probability of having an attack in the process in a way that can compromise it is as small as possible.

Elliptic curves have some properties, in  this case, for example:

\begin{itemize}
    \item A non-vertical line that intersects two non-tangent points on the curve will always intersect a third point on the same curve.
    \item A non-vertical line tangent to the curve at one point will intersect only one other point on the curve.
\end{itemize}

This properties allow us to define two operations: point addition and point doubling.

\paragraph{Point addition}

\tab Point addition consists of following this steps:

\begin{itemize}
    \item Take two points of the elliptic curve, \(P\) and \(Q\).
    \item Pass a line through the previous points.
    \item Mark the point (\(-R\)) from the intersection of the previous line with the curve.
    \item Mirror \(-R\) through the x-axis, obtaining \(R\).
\end{itemize}

This algorithm can be put into the following expression:

\[P + Q = R\]

With \(R\) being the result of the sum of the original points.

Finally, this equation can be translated to the following graphic:

\begin{figure}[H]
    \begin{center}
        \includegraphics[width=0.5 \textwidth]{images/point_addition.png}
        \caption{\textit{Point addition}}
    \end{center}
\end{figure}

\paragraph{Point doubling}

\tab Point doubling consists of following this steps:

\begin{itemize}
    \item Take a point of the elliptic curve, \(P\).
    \item Pass a tangent line tangent to \(P\).
    \item Mark the point (\(-R\)) from the intersection of the previous line with the curve.
    \item Mirror \(-R\) through the x-axis, obtaining \(R\).
\end{itemize}

This algorithm can be put into the following expression:

\[P + P = 2P = R\]

With \(R\) being the result of the sum of \(R\) with itself.

Finally, this equation can be translated to the following graphic:

\begin{figure}[H]
    \begin{center}
        \includegraphics[width=0.5 \textwidth]{images/point_doubling.png}
        \caption{\textit{Point doubling}}
    \end{center}
\end{figure}

\paragraph{Scalar multiplication}

\tab Point addition and point doubling combined are used for scalar multiplication, this means adding a point to itself \(n\) times:

\[R = nP\]

To explain it better, we will use the following example:

\[R = 11P\]
\[R = P + (P + (P + (P + (P + (P + (P + (P + (P + (P + P)))))))))\]

This can be simplified using the previous operations:

\[R = 11P\]
\[R = P + 10P\]
\[R = P + 2(5P)\]
\[R = P + 2(P + 4P)\]
\[R = P + 2(P + 2(2P))\]

As we can see, we just decomposed \(11P\) into two point additions and three point doubling. This can be very useful when \(n\) is a large number.

One of the uses of scalar multiplication is to calculate the private-public key pair, we will take a look on how it works and where it is placed in the Bitcoin protocol in the section bellow.

\subsubsection{Private-Public key pair generation}

\tab The private-public key pair is used on Bitcoin transactions between users. This keys can be represented as:

\begin{itemize}
    \item \textbf{Public key} - represents a public Bitcoin address
    \item \textbf{Private key} - personal key, secret and unique to each user.
\end{itemize}

For generating this pair, we first need a initial point, usually referred as generator point. This point multiplied by the private key will give us the public key:

\[public\: key = generator\: point \times private\: key\]

This is the same as saying that we will add the point to itself as many times as the private key value.

Graphically is easier to understand this. For instance, if we take a point \(P\) of the \textit{secp256k1} curve, and we say that the private key is 4, we will add P to itself 4 times, obtaining \(4P\), this would be the value of the public key associated with the given private one. The process would result in this graph:

\begin{figure}[H]
    \begin{center}
        \includegraphics[width=0.5 \textwidth]{images/Kobiltz_curve_with_points.png}
        \caption{\textit{Process of obtaining \(4P\) from the point \(P\), using scalar multiplication}}
    \end{center}
\end{figure}

And this is only for a private key of 4, if the private key is a larger number, there are many more calculations to be made, and many other points created during this process. 

As we previously discussed, the curve used for public-private key generation in the case of Bitcoin has a parameter \(p\), that is the prime order. This number, being a large prime number, prevents the points taking decimal numbers, during the scalar multiplication process, making the curve a collection of dots.

On the graph bellow, we can see a "curve" of finite field with \(b = 7\) and \(p = 37\):

\begin{figure}[H]
    \begin{center}
        \includegraphics[width=0.6 \textwidth]{images/finite_field_b7_p37.png}
        \caption{\textit{"Curve" of finite field with prime order 37}}
    \end{center}
\end{figure}

Doing this calculation on today's computers is rather fast and efficient, but reversing the same operation is very complex. Although deducing the private key from the public key is possible, it would take a huge time to do so, this process would take \(2^{128}\) attempts, meaning that, even with a million CPU\footnote{Central Processing Unit}s, it would take approximately 260 billion times the age of the universe to break Bitcoin's elliptic curve cryptography.

Generated the pair, this is now used to encrypt data used on Bitcoin transactions.






% Add "References" to table of contents
\addcontentsline{toc}{section}{References}
% No cite makes all references appear, even if there's no citation on the text
\nocite{*}
\printbibliography

\end{document}