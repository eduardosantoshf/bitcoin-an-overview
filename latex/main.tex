\documentclass{article}
% Margin definition.
\usepackage[a4paper,total={6.8in, 8.5in}]{geometry}
\usepackage{parskip}
% Images.
\usepackage{graphicx}
\usepackage[hidelinks, bookmarks=true]{hyperref}
\usepackage{float}
% Encoding.
\usepackage[english]{babel}
\usepackage[utf8]{inputenc}
% Allow multiline comments
\usepackage{verbatim} 
% To have another layer of sub sections - \paragraph
\usepackage{titlesec}

\setcounter{secnumdepth}{4}

\titleformat{\paragraph}
{\normalfont\normalsize\bfseries}{\theparagraph}{1em}{}
\titlespacing*{\paragraph}
{0pt}{3.25ex plus 1ex minus .2ex}{1.5ex plus .2ex}
% Helvetic font.
\usepackage[scaled]{helvet}
\renewcommand\familydefault{\sfdefault} 
% Header for UA logo.
\usepackage{fancyhdr}
% package for plots / graphics
\usepackage{pgfplots}
\pgfplotsset{width=10cm,compat=newest}
% Dots in index.
\usepackage[titles]{tocloft}
\renewcommand{\cftsubsecleader}{\Large\cftdotfill{0}}
\renewcommand{\cftsecleader}{\Large\cftdotfill{0}}
\renewcommand{\cftsecfont}{\large\bfseries\scshape}
\renewcommand{\cftsubsecfont}{\scshape}
\renewcommand*{\HyperDestNameFilter}[1]{\jobname-#1}
% Dot after number in (sub)sections and in toc.
\renewcommand{\cftsecaftersnum}{.}
\renewcommand{\cftsubsecaftersnum}{.}
\usepackage{titlesec}
\titlelabel{\hspace{-0.5cm}\quad}
\usepackage[letterspace=45]{microtype}
\newcommand*{\fullref}[1]{\hyperref[{#1}]{\autoref*{#1} \nameref*{#1}}}
% Header with UA logo definition. 
\pagestyle{fancy}
\fancyhf{}
\chead{
    \includegraphics[width=5in]{./images/header_ua.png}
}
\setlength\headheight{20pt}
% Footer with page number.
\rfoot{Page \thepage}
\renewcommand{\footrulewidth}{0.1pt}
% Rename table of contents title to "Index"
\renewcommand{\contentsname}{\normalsize Index \vspace{0.6cm}}
% Add text with hyperlink
\usepackage{hyperref}
%\hypersetup{
%    colorlinks=true,
%    linkcolor=blue,
%    filecolor=magenta
%}
% Water mark
\newsavebox\mybox
\usepackage[printwatermark]{xwatermark}
\usepackage{xcolor}
\usepackage{tikz}
% paragraph
\newcommand\tab[1][1cm]{\hspace*{#1}}
\setlength\parindent{24pt}
%images
 \usepackage{graphicx}
\usepackage{caption}
% footnotes at bottom
\usepackage[bottom]{footmisc}
% Urls with line break
\usepackage{pdflscape}
% Drawing functions
\usepackage{tikz}
\usepackage{pgfplots}
\pgfplotsset{width=7cm, height=4cm, compat=1.17}

\usepackage{multicol}
\setlength{\columnsep}{1cm}

%%%%%%%%%%%% References/Bibliography %%%%%%%%%%%%
\usepackage{biblatex}
\addbibresource{bibliography.bib}

%%%%%%%%%%%%%%%%%%%%%%%%%%%%%%%%%%%%%%%%%%%%%%%%%

\begin{document}

%%%%%%%%%%%%%%%%%% Cover Page %%%%%%%%%%%%%%%%%%
\title{\vspace{-0.9cm}
       \vspace{1cm}
       \normalsize
       \raggedright\textbf{Title: \hspace{1.5cm} Bitcoin: An Overview} \\ \vspace{0.4cm}
       \raggedright\textbf{Authors: \hspace{0.95cm} Eduardo Santos, Hugo Ferreira, João Soares, Pedro Bastos} \\ \vspace{0.4cm}
       \raggedright\textbf{Date: \hspace{1.45cm} 13/04/2021} \\}
\author{}
\date{}

\maketitle
\thispagestyle{fancy}

%%%%%%%%%%%%%%%%%% END Cover Page %%%%%%%%%%%%%%%%%%

\vspace{-1.4cm}

\tableofcontents


\fontsize{10pt}{13pt}
\selectfont
\lsstyle

\titlelabel{\thetitle.\quad}	

\newpage

\section{Introductory Note}

\tab This report consists of a brief analysis of Bitcoin and its protocol, exploring its surge and growth in the cryptocurrency market.
We will also address the topic of Blockchain, and how does it work.

\section{Summary / Abstract}

\tab Bitcoin is a cryptocurrency that has been widely spoken over the past few years, majorly because of what it represents: the beginning if the globalization of digital currencies.

\section{Cryptocurrencies}

\section{Blockchain}

\subsection{What is the Blockchain?}

\subsection{How does it work?}

\section{Bitcoin}

\subsection{What is Bitcoin?}

\subsection{The Seven Pillars of Bitcoin}

\subsection{The Bitcoin Protocol}

\tab Being a decentralized digital currency, Bitcoin is not something that we can physically own, which in itself generates a large deviation from the concept of money that we are used to and that we use in our day-to-day. To use this digital asset, there are many steps/rules that need to be followed, they make up the Bitcoin protocol.

Bitcoins are not stored centrally nor locally, they exist in a distributed ledger called blockchain. This ledger runs on a P2P\footnote{peer-to-peer} network of computers. 

Owning a Bitcoin is, nothing more, nothing less than simply having the power to transfer it to someone else, with the transaction being recorded in the blockchain. But how does this work?

This is made possible with the usage of an private key - public key ECDSA\footnote{Elliptic Curve Digital Signature Algorithm} pair.

\subsubsection{ECDSA}

\tab ECDSA is a variant of the DSA\footnote{Digital Signature Algorithm}, this uses elliptic curve cryptography. An elliptic curve is mathematically represented by the following equation:

\[y^2 = ax^3 + bx + c\]

In the case of Bitcoin:

\[a = 1\]
\[b = 0\]
\[c = d\: mod\: p\]
\[d = 7\]
\[p = 1.158 \times 10^{77}\]

Replacing the values, we obtain:

\[y^2 = x^3 + 7 \bmod{1.158 \times 10^{77}}\]

Finally, this equation can be translated to the following graphic:

\vspace{5mm} %5mm vertical space

%Added [H] after so that image is placed after the text above
%This is only possible by adding also \usepackage{float} to the preamble
\begin{figure}[H]
    \begin{center}
        \includegraphics[width=0.6 \textwidth]{images/Kobiltz_curve.png}
        \caption{\textit{Kobiltz curve}}
    \end{center}
\end{figure}

\vspace{5mm} %5mm vertical space

\begin{comment}
%Added [H] after so that image is placed after the text above
%This is only possible by adding also \usepackage{float} to the preamble
\begin{figure}[H]
    \begin{center}
        \begin{tikzpicture}
            \begin{axis}[
                width=0.5\textwidth,
                height=0.3\textwidth,
                xmin=-5,
                xmax=5,
                ymin=-5,
                ymax=5,
                xlabel={$x$},
                ylabel={$f(x)$},
                scale only axis,
                axis lines=middle,
                smooth,
                domain=-1.912931:3, %supostamente, sem esta linha o gráfico divide-se em 2 partes, tristeza
                xtick={-5,...,0,...,5},
                ytick={-5,...,0,...,5},
            ]
                \addplot [red] {sqrt(x^3 + mod(7, 1.158*10^(77)))};
                \addplot [red] {-sqrt(x^3 + mod(7, 1.158*10^(77)))};
            \end{axis}
        \end{tikzpicture}
    \end{center}
    \caption{Kobiltz curve}
\end{figure}
\end{comment}

The curve on the graph above is called \textit{Kobiltz curve}, and its often referred as \textit{secp256k1}. This curve was created on purpose, and with a very defined objective: to increase efficiency in the generation of public and private keys, making the computation of this key pair 30\% faster. But this is not the only thing that differentiates this curve from the rest of the elliptical curves, due to the precision in choosing the values of the constants \(a, b, c\) and \(d\), the probability of having an attack in the process in a way that can compromise it is as small as possible.

Elliptic curves have some properties, in  this case, for example:

\begin{itemize}
    \item A non-vertical line that intersects two non-tangent points on the curve will always intersect a third point on the same curve.
    \item A non-vertical line tangent to the curve at one point will intersect only one other point on the curve.
\end{itemize}

This properties allow us to define two operations: point addition and point doubling.

\paragraph{Point addition}

\tab Point addition consists of following this steps:

\begin{itemize}
    \item Take two points of the elliptic curve, \(P\) and \(Q\).
    \item Pass a line through the previous points.
    \item Mark the point (\(-R\)) from the intersection of the previous line with the curve.
    \item Mirror \(-R\) through the x-axis, obtaining \(R\)
\end{itemize}

This algorithm can be put into the following expression:

\[P + Q = R\]

With \(R\) being the result of the sum of the original points.

Finally, this equation can be translated to the following graphic:

\vspace{5mm} %5mm vertical space

%Added [H] after so that image is placed after the text above
%This is only possible by adding also \usepackage{float} to the preamble
\begin{figure}[H]
    \begin{center}
        \includegraphics[width=0.6 \textwidth]{images/point_addition.png}
        \caption{\textit{Point addition}}
    \end{center}
\end{figure}

\vspace{5mm} %5mm vertical space

\paragraph{Point doubling}

\tab Point doubling consists of following this steps:

\begin{itemize}
    \item Take a point of the elliptic curve, \(P\).
    \item Pass a tangent line tangent to \(P\).
    \item Mark the point (\(-R\)) from the intersection of the previous line with the curve.
    \item Mirror \(-R\) through the x-axis, obtaining \(R\)
\end{itemize}

This algorithm can be put into the following expression:

\[P + P = 2P = R\]

With \(R\) being the result of the sum of \(R\) with itself.

Finally, this equation can be translated to the following graphic:

\vspace{5mm} %5mm vertical space

%Added [H] after so that image is placed after the text above
%This is only possible by adding also \usepackage{float} to the preamble
\begin{figure}[H]
    \begin{center}
        \includegraphics[width=0.6 \textwidth]{images/point_doubling.png}
        \caption{\textit{Point doubling}}
    \end{center}
\end{figure}

\vspace{5mm} %5mm vertical space

\paragraph{Scalar multiplication}

\tab Point addition and point doubling combined are used for scalar multiplication, this means adding a point to itself \(n\) times:

\[R = nP\]

To explain it better, we will use the following example:

\[R = 11P\]
\[R = P + (P + (P + (P + (P + (P + (P + (P + (P + (P + P)))))))))\]

This can be simplified using the previous operations:

\[R = 11P\]
\[R = P + 10P\]
\[R = P + 2(5P)\]
\[R = P + 2(P + 4P)\]
\[R = P + 2(P + 2(2P))\]

As we can see, we just decomposed \(11P\) into two point additions and three point doubling. This can be very useful when \(n\) is a large number, and scalar multiplication itself will be needed for calculating the private-public key pair.

%We already mentioned the private key - public key pair, but what is it and what is it for? Let's %take a look at it in the section below.

%\subsubsection{Private-Public key pair}

%\tab In this case, this pair is used in bitcoin and other cryptocurrencies transactions.

% Add "References" to table of contents
\addcontentsline{toc}{section}{References}
% No cite makes all references appear, even if there's no citation on the text
\nocite{*}
\printbibliography

\end{document}